\documentclass[a4paper]{book}
\usepackage{amsmath}
\begin{document}
Plain aligns

\begin{align}
A =& B \\
C =& D \\
\end{align}

\begin{align}
A =& B \\
C =& D \\
&+E \notag
\end{align}

Equations \eqref{eq:before} and \eqref{eq:after}:
\begin{align}
\label{eq:before}
&x \quad c_1,
\end{align}
and
\begin{align}
&y \quad c_2,
\label{eq:after}
\end{align}

Aligned within math
\[\begin{aligned}
    A =& B
  \end{aligned}
\]

Tendency to see the ampersand before aligned has started
\[
C =\left\{\begin{aligned}
& D \end{aligned}\right.
\]

Align with aligned
\begin{align*}
A & B\\
C &\left\{\begin{aligned}
& D \end{aligned}\right.
\end{align*}

\begin{align}
    A =& \left\{
\begin{aligned}
    B & \text{if}\\
    C & \text{otherwise}
  \end{aligned}\right.
 \end{align}
Or even
\begin{align}
\begin{aligned}
& a\\
& b
\end{aligned}
\end{align}
 
Equation with split; Ampersand before relation
\begin{equation}
\begin{split}
C &= D \\
&{}+E
\end{split}
\end{equation}
or after
\begin{equation}
\begin{split}
C =& D \\
&{}+E
\end{split}
\end{equation}

Gather with split
\begin{gather}
A = B \\
\begin{split}
C &= D \\
&{}+E
\end{split}\\
 F = G
\end{gather}

Align with split.
Notice that the split acts like a column pair in the align.
\begin{align}
A &= B \\
\begin{split}
C &= D \\
&{}+E
\end{split}\\
 F &= G\\
H &= I
\end{align}
\begin{align}
A =&\ B \\
B=& - C \\
\begin{split}
C =& -D + E + F + G + H\\
&+I
\end{split}
\end{align}

2-Column align, with missing columns
\begin{align}
  A &= B &
 C &= D \\
  E &= F \\
  H &=  I &
  J &= K
\end{align}

2-Column align with split;
Note that you should omit double slash from last line of split!
\begin{align}
  A &= B &
 C &= D \\
\begin{split}
  E &= F \\
    &{} + G
\end{split}\\
  H &=  I &
  J &= K
\end{align}

\begin{align}
  A &= B &
  C &= D \\
  X &= Y &
\begin{split}
  E &= F \\
    &{} + G
\end{split}\\
  H &=  I & J &= K \\
  \begin{split}
    L &= M\\
    &{}+N
  \end{split} &
 O &= P\\
  Q &= R &
  S &= T
\end{align}

Multiple splits in multicolumn align.
Note how each split block contributes to a single row, but it is
horizontally aligned as if it were a column pair in the align.
\begin{align}
  A &= B+b+c+d &
  C &= D \\
\begin{split}
  E+e+f+g &= F \\
    &{} + G
\end{split} &
\begin{split}
    L &= M\\
    &{}+N \\
    &{}+Z
  \end{split} \\
  Q &= R &
  S &= T
\end{align}

%\end{document}
%\end{document}




\end{document}
