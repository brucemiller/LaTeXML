\documentclass{article}
\begin{document}
\section{Version}
Running eTeX version \the\eTeXversion\eTeXrevision.

\section{Old style}
\def\baz{BAZ!}
baz \expandafter\ifx\csname baz\endcsname\relax IS \else is NOT \fi relax;
baz \ifx\baz\undefined IS equal \else is NOT equal \fi undefined;
baz \ifdefined\baz IS \else is NOT \fi defined.

foo \expandafter\ifx\csname foo\endcsname\relax IS \else is NOT \fi relax;
foo \ifx\foo\undefined IS equal \else is NOT equal \fi undefined;
foo \ifdefined\foo IS \else is NOT \fi defined.

\section{New style}
\def\ebaz{BAZ!}
ebaz \ifcsname ebaz\endcsname IS \else is NOT \fi bound;
ebaz \ifx\ebaz\undefined IS equal \else is NOT equal \fi undefined;
ebaz \ifdefined\ebaz IS \else is NOT \fi defined.

efoo \ifcsname efoo\endcsname IS \else is NOT \fi bound;
efoo \ifx\efoo\undefined IS equal \else is NOT equal \fi undefined;
efoo \ifdefined\efoo IS \else is NOT \fi defined.

\section{Expansion}
Regular form:
\def\When{Now}%
\edef\Expanded{Was expanded \When}%
\def\When{Later}%
Got \Expanded.

With noexpand inserted:
\def\When{Now}%
\edef\Expanded{Was expanded \noexpand\When}%
\def\When{Later}%
Got \Expanded.

Using the new protected primitive:
\protected\def\eWhen{Now}%
\edef\Expanded{Was expanded \eWhen}%
\protected\def\eWhen{Later}%
Got \Expanded.

\section{Tokens}
\detokenize{This_and^that}.

\catcode`*=\active
\def*{STAR}
A Shooting *.

\def\Stuff{\unexpanded{A Falling *}.}
\Stuff
\def*{from grace}
\Stuff

\section{Unless}
\newcount\one\one=1\relax
\newcount\two\two=2\relax
\newcount\many\many=99\relax

\newif\ifalwaystrue\alwaystruetrue
\newif\ifalwaysfalse\alwaysfalsefalse

\ifnum\one>\two 1 greather than 2 \else not 1 greater than 2\fi.
\ifnum\many>\two many greater than 2 \else not many greater than 2\fi.

True \ifalwaystrue is always true \else should never be false\fi.
False \ifalwaysfalse should never be true \else is always false\fi.

\ifodd\one one is odd\else one is not odd\fi.
\ifodd\two two is odd\else two is not odd\fi.

one is \ifcase\one zero \or one \or two \or three \else many\fi.
two is \ifcase\two zero \or one\or two\or three \else many\fi.
many is \ifcase\many zero \or one\or two\or three \else many\fi.


INVERT \unless\ifnum\one>\two 1 greater than 2 \else not 1 greater than 2\fi.
INVERT \unless\ifnum\many>\two many greater than 2 \else not many greater than 2\fi.

INVERT \unless\ifodd\one one is odd\else one is not odd\fi.
INVERT \unless\ifodd\two two is odd\else two is not odd\fi.

INVERT True \unless\ifalwaystrue is always true \else should never be false\fi.
INVERT False \unless\ifalwaysfalse should never be true \else is always false\fi.

% You cannot use \unless before \ifcase!
% You cannot use \unless before a macro (even if it expands to \if...)

\section{Registers}
\newbox\mybox
\newdimen\mydima
\newdimen\mydimb
\def\mywidth{\wd\mybox}
\protected\def\protectedmywidth{\wd\mybox}
\setbox\mybox\hbox{Some Text}
\the\mywidth

\mydima\mywidth
Width:\the\mydima

\mydimb\protectedmywidth
Width:\the\mydimb

%\input{etex-mydim}\mywidth
%Width:\the\mydima

\def\mypts{pt}
\def\myminus{-}
\def\myone{1}
\def\mytwo{2}
\mydima\myminus\myone\mytwo\mypts\relax
Width:\the\mydima

\protected\def\protectedmypts{pt}
\protected\def\protectedmyminus{-}
\protected\def\protectedmyone{1}
\protected\def\protectedmytwo{2}
\mydima\protectedmyminus\protectedmyone\protectedmytwo\protectedmypts\relax
Width:\the\mydima

\subsection{Parshape}
Parshape is sorta like a register
\newcount\mynum
\parshape 2 1pt 2pt 3pt 4pt
\mynum=\parshape\relax \the\mynum lines:
Indents:\the\parshapeindent-1\relax,
\the\parshapeindent1\relax,
\the\parshapeindent2\relax,
\the\parshapeindent3\relax;
Lengths:\the\parshapelength-1\relax,
\the\parshapelength1\relax,
\the\parshapelength2\relax,
\the\parshapelength3\relax;
Dimens:
\the\parshapedimen-2\relax,
\the\parshapedimen-1\relax,
\the\parshapedimen1\relax,
\the\parshapedimen2\relax,
\the\parshapedimen3\relax,
\the\parshapedimen4\relax;
\the\parshapedimen5\relax;
\the\parshapedimen6\relax;

\subsection{Penalties}
\interlinepenalties 2 10 20
Interline penalties:
 \mynum=\interlinepenalties 1\relax \the\mynum,
 \mynum=\interlinepenalties 2\relax \the\mynum,
 \mynum=\interlinepenalties 3\relax \the\mynum.

Interline penalties:
 \mynum=\interlinepenalties 1\relax \the\mynum.

\interlinepenalties 2 10 20
\interlinepenalties 0
Interline penalties:
 \mynum=\interlinepenalties 1\relax \the\mynum.

 \clubpenalties 2 10 20
Club penalties:
 \mynum=\clubpenalties 1\relax \the\mynum,
 \mynum=\clubpenalties 2\relax \the\mynum,
 \mynum=\clubpenalties 3\relax \the\mynum.
 
 Club penalties:
 \mynum=\clubpenalties 1\relax \the\mynum.

\clubpenalties 0
Club penalties:
 \mynum=\clubpenalties 1\relax \the\mynum.

widowpenalties, displaywidowpenalties should behave the similarly.

\section{Glue}
\def\showskip#1{
  \the#1 : stretch \the\gluestretch#1\ \the\gluestretchorder#1 ;
 shrink \the\glueshrink#1\ \the\glueshrinkorder#1.}
\newskip\myskip

\myskip=10pt plus 1fil\relax
\showskip\myskip

\myskip=10pt minus 2fill\relax
\showskip\myskip

\myskip=10pt plus 1pt minus 2pt\relax
\showskip\myskip


\section{Char sizes}
\def\charsize#1#2{
  \the\fontcharwd#1 `#2
\ x \the\fontcharht#1 `#2
\ + \the\fontchardp#1 `#2
\ c \the\fontcharic#1 `#2}

\font\tenrm=cmr10
Size of x: \charsize\tenrm x.
Size of X: \charsize\tenrm X.

\end{document}