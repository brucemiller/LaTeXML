\documentclass{article}
\begin{document}
\section{Testing various objects with ifx}
\def\testA{foo}
\def\testB{foo}
\def\testC{bar}

Test Letters A and A (True) : \ifx AA True \else False\fi.

Test Letters A and B (False) : \ifx AB True \else False\fi.

Test Macros testA and testB (True) : \ifx\testA\testB True \else False\fi.

Test Macros testA and testC (False) : \ifx\testA\testC True \else False\fi.

Nested (True False)
\ifx\testA\testB True \ifx\testA\testC True \else False\fi \else False \ifx\testA\testC True \else False\fi\fi.
Nested (False True)
\ifx\testA\testC True \ifx\testA\testB True \else False\fi \else False \ifx\testA\testB True \else False\fi\fi.

\let\endif\fi
Test let fi (True False); 
\ifx\testA\testB True \ifx\testA\testC True \else False\endif \else False\ifx\testA\testC True \else False\endif\endif.

\let\otherwise\else
Test let else (True False); 
\ifx\testA\testB True \ifx\testA\testC True \otherwise False\endif \otherwise False\ifx\testA\testC True \otherwise False\endif\endif.


ifx does NOT expand conditional tokens (fooFalse): \ifx\iftrue\testA True\else\testB False\fi.

else is NOT expandable while expanding the test clause

\section{Testing numerics ifnum and ifcase}
\newcount\c\c=1
OK OK :
\ifnum\c>2\else OK \fi
\ifodd18\else OK\fi.

\def\oneorten{1\ifodd\c\else 0\fi}
OK :
\c=1\relax\ifnum\oneorten>5\else OK \fi
\c=2\relax\ifnum\oneorten>5\else NOT OK\fi


\newcount\negone\negone=-1\relax
\newcount\zero\zero=0\relax
\newcount\one\one=1\relax
\newcount\two\two=2\relax
\newcount\many\many=99\relax
negone is \ifcase\negone zero\or one\or two\or three\else other\fi.
zero is \ifcase\zero zero\or one\or two\or three\else other\fi.
one is \ifcase\one zero\or one\or two\or three\else other\fi.
two is \ifcase\two zero\or one\or two\or three\else other\fi.
many is \ifcase\many zero\or one\or two\or three\else other\fi.

\section{Testing if}
Test plain if, as well;

Test a and a \if aa True\else False\fi

Test A and a \if Aa True\else False\fi

Test f and testA \if f\testA True\else False\fi

Test f and testC \if f\testC True\else False\fi

Test b and testC \if f\testC True\else False\fi

Test testc and b  \if f\testC True\else False\fi

\c=1\relax Test 1 and oneorten [c=1]  \if 1\oneorten True\else False\fi

\c=2\relax Test 1 and oneorten [c=2]  \if 1\oneorten True\else False\fi

\section{Testing if in test clause}
\def\testifx{T\ifx\testA\testB T\else F\fi}

Test ifx testA testB \testifx

Test if on testifx \if\testifx True\else False\fi

Test ifx on testifx \if\testifx True\else False\fi

\def\nottestifx{T\ifx\testA\testB F\else T\fi}

Test not ifx testA testB \nottestifx

Test if on nottestifx \if\nottestifx True\else False\fi

Test ifx on nottestifx \ifx\nottestifx True\else False\fi


\def\a{a}%
 \let\b=\a
 \def\check{T\ifx\a\b F\else T\fi}

 % Test the check value:
 \check\\

 % Test it used in \if:
 if-check is: \if\check true\else false\fi


\def\innerif{\if\clause X\else Y\fi}

True: \def\clause{TT}\if X\innerif True\else False\fi

False: \def\clause{TF}\if X\innerif True\else False\fi

False: \def\clause{TT}\if Y\innerif True\else False\fi

True: \def\clause{TF}\if Y\innerif True\else False\fi

% TeXbook, exercise 20.11, test \if with \let
\def\a{*}
\let\b=*
\def\c{/}

Expecting (True True False):
\if*\a True \else False \fi
\if\a\b True \else False \fi
\if\a\c True \else False \fi

However, ifx doesn't expand, and if's can be tested!
True: \ifx\iftrue\iftrue True\else False\fi;
False: \ifx\iftrue\iffalse True\else False\fi.
\end{document}
