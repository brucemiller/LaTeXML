
Testing low-level \TeX\ font manipulations.

\def\testfont#1{%
  \par -- Font \string#1(\fontname#1; \meaning#1):{X{\the#1X}X}; hyphen=\the\hyphenchar#1.}

Stock plain fonts:
\testfont\font
\testfont\fiverm
\testfont\tenrm

New 5pt fonts, new default {\tt hyphenchar=99}.
{\bf All} shared; {\tt scaled} mapped to {\tt at}:
\defaulthyphenchar=99
\font\myrmfiveA=cmr10 at 5pt
\font\myrmfiveB=cmr10 at 5pt
\font\myrmfiveC=cmr10 scaled 500
\testfont\myrmfiveA
\testfont\myrmfiveB
\testfont\myrmfiveC

Bump {\tt hyphenchar=100} on myrmfiveA:
\hyphenchar\myrmfiveA=100
\testfont\myrmfiveA
\testfont\myrmfiveB
\testfont\myrmfiveC

Bump {\tt hyphenchar=101} on myrmfiveA, {\it as if} grouped:
{\hyphenchar\myrmfiveA=101}
\testfont\myrmfiveA
\testfont\myrmfiveB
\testfont\myrmfiveC

While stock fonts should be unchanged
\testfont\fiverm
\testfont\tenrm


\string\font\ is weird;
it snapshots of the current font;
but compare meanings:
{\fiverm\expandafter\let\expandafter\xxxrm\the\font \global\let\xxxrm\xxxrm}
{\fiverm\testfont\font}
\testfont\fiverm
%{\expandafter\let\expandafter\xxxrm\the\font X\fiverm X \xxxrm X;

%  But compare meanings: \meaning\font\ vs \meaning\xxxrm.
%  }

\testfont{\textfont2}

Testing plain math fonts:
\def\sample{abc123}
\def\tester#1{%
  \par{\string#1: {#1 \sample} and ${#1 \sample}$}}

\def\noop{}
\tester\noop
\tester\rm
\tester\mit
\tester\cal
\tester\it
\tester\sl
\tester\bf
\tester\tt

Testing Text glyph lookup.
\font\tenbsy=cmbsy10
\font\tenmsa=msam10
\font\tenmsb=msbm10

Normal:
{\tenrm\char"10};{\tenrm\char"41}.
{\teni\char"10};{\teni\char"41}.
{\tenbf\char"10};{\tenbf\char"41}.
{\tentt\char"10};{\tentt\char"41}.
{\tensl\char"10};{\tensl\char"41}.
{\tenit\char"10};{\tenit\char"41}.

Symbol:
{\tensy\char"10};{\tensy\char"41}.
{\tenbsy\char"10};{\tenbsy\char"41}.
{\tenex\char"10};{\tenex\char"41}.

AMS:
{\tenmsa\char"10};{\tenmsa\char"41}.
{\tenmsb\char"10};{\tenmsb\char"41}.

\bye
