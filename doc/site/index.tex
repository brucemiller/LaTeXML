\documentclass{article}
\usepackage{latexml}
\usepackage{hyperref}
\usepackage{../sty/latexmldoc}
\usepackage{listings}
% Should the additional keywords be indexed?
\lstdefinestyle{shell}{language=bash,escapechar=@,basicstyle=\ttfamily\small,%
   morekeywords={latexml,latexmlpost,latexmlmath},
   moredelim=[is][\itshape]{\%}{\%}}

\input{releases.tex}
\newcommand{\PDFIcon}{\includegraphics{pdf}}

\title{\LaTeXML\ \emph{A \LaTeX\ to XML Converter}}
\begin{document}
\maketitle

%============================================================
The \emph{Current release} is \htmlref{\CurrentVersion}{get}.

In the process of developing the
\href{http://dlmf.nist.gov/}{Digital Library of Mathematical Functions},
we needed a means of transforming
the LaTeX sources of our material into XML which would be used
for further manipulations, rearrangements and construction of the web site.
In particular, a true `Digital Library' should focus on the \emph{semantics}
of the material, and so we should convert the mathematical material into both
content and presentation MathML.
At the time, we found no software suitable to our needs, so we began
development of LaTeXML in-house.  

In brief, \texttt{latexml} is a program, written in Perl, that attempts to
faithfully mimic \TeX's behaviour, but produces XML instead of dvi.
The document model of the target XML makes explicit the model implied
by \LaTeX.
The processing and model are both extensible; you can define
the mapping between \TeX\ constructs and the XML fragments to be created.
A postprocessor, \texttt{latexmlpost} converts this
XML into other formats such as HTML or XHTML, with options
to convert the math into MathML (currently only presentation) or images.

\emph{Caveats}: It isn't finished, there are gaps in the coverage,
particularly in missing implementations of the many useful \LaTeX\ packages.
But is beginning to stabilize and interested parties
are invited to try it out, give feedback and even to help out.


%============================================================
\section{Examples}\label{examples}
At the moment, the best example of \LaTeXML's output is 
the \href{http://dlmf.nist.gov/}{DLMF} itself
--- the preview  will soon contain several chapters.
It is perhaps not a fair example, in that there is much
special-purpose processing to generate cross-referencing and metadata that is not
part of LaTeXML by default. Nevertheless, it gives an indication of the
math conversion and other features possible.

\begin{description}
\item[\href{http://dlmf.nist.gov/}{DLMF}]
   The Digital Library of Mathematical Functions was the
   primary instigator for this project.
\item[\href{manual/}{LaTeXML Manual}]
   The \LaTeXML\ User's manual (\href{manual.pdf}{\PDFIcon}).
\item[\href{examples/tabular/tabular.html}{LaTeX tabular}]
    from the \LaTeX\ manual, p.205.
    (\href{examples/tabular/tabular.tex}{\TeX},
     \href{examples/tabular/tabular.pdf}{\PDFIcon})
\item[These pages] were produced using \LaTeXML, as well.
\end{description}


%============================================================
\section{Get \LaTeXML}\label{get}
\paragraph{Current Release:}\label{download.current}
The current release is \textbf{\CurrentVersion}
(also see the \href{Changes}{Change Log}).

\LaTeXML\ itself is written in `pure' Perl, but requires \TeX,
and several perl modules, some of which depend on C libraries --- the \ref{prerequisites}.
It should always be possible to install \LaTeXML\ on any platform
with a sufficiently recent \href{http://www.perl.org/}{Perl},
by following the instructions at \ref{install.general}.

%In most cases, however, it will be preferable to install
%a platform-specific, pre-built, release, as this often
%takes care of installing dependencies and updates,
%and allows cleaner uninstall.  

\emph{Normally}, it would be preferable to install a platform-specific, prebuilt release.
But frankly, the released version is rather old;
so out-of-date, in fact, that we are inclined to wait until we
have a really dramatic replacement before we make a new formal release.
Until then, you will probably be best served by \ref{install.svn}.

In any case, it is usually preferable to use a platform-specific procedure
for installing the prerequisites, and there may be platform-specific
instructions regarding SVN.
Thus, please consult \ref{install.osnotes}, first,
for platform-specific releases and instructions.

As a rule of thumb, any `recent' version of any prerequisite software will
work with \LaTeXML, something released within the last few years;
Perl should be at least version 5.8.
Of course for security reasons, you should update all software on your system
regularly to install patches and fix any bugs.
When maintaining your system using a platform specific software
updating tool, such as \texttt{yum}, \texttt{apt} or \texttt{port},
you should regularly use the appropriate `update' command.

\subsection{General Installation Procedures}\label{install.general}
The following sections outline the general procedures for installing
\LaTeXML\ and the supporting libraries and modules that it needs.

\subsubsection{Installing Prerequisites}\label{install.prerequisites}
\begin{itemize}
\item If \texttt{libxml2} and \texttt{libxslt} are are not already installed:\\
  Follow the instructions at \href{http://www.xmlsoft.org}{XMLSoft} to
  download and install the most recent versions of \texttt{libxml2} and \texttt{libxslt}.
\item If the Perl modules \texttt{Parse::RecDesent}, \texttt{XML::LibXML}
 and \texttt{XML::LibXSLT} are not already installed:\\
 Use CPAN to install the required perl modules (typically as root):
\begin{lstlisting}[style=shell]
   perl -MCPAN -e shell
   cpan> install DB_File, Parse::RecDescent
   cpan> install XML::LibXML, XML::LibXSLT
   cpan> quit
\end{lstlisting}
\item If \texttt{ImageMagick} and its Perl binding are not already installed:\\
  Follow the instructions at \href{http://www.imagemagick.org/}{ImageMagick}
  to download and install the latest version of ImageMagick being sure to enable
  and build the Perl binding along with it.
\end{itemize}

\subsubsection{Installing from Source}\label{install.source}
\begin{itemize}
\item Download \CurrentTarball
\item Unpack and Build \LaTeXML, following the usual Perl module incantations:
\begin{lstlisting}[style=shell]
   tar zxvf LaTeXML-@\CurrentVersion@.tar.gz
   cd LaTeXML-@\CurrentVersion@
   perl Makefile.PL
   make
   make test
\end{lstlisting}
\item Install, as root:
\begin{lstlisting}[style=shell]
   make install
\end{lstlisting}
\end{itemize}
(See \texttt{perl perlmodinstall} for more details, if needed.)

You can specify nonstandard place to install files (and avoid the need to install as root)
by modifying the Makefile creating command above to
\begin{lstlisting}[style=shell]
   perl Makefile.PL PREFIX=%perldir% TEXMF=%texdir%
\end{lstlisting}
where \emph{perldir} is where you want the perl related files to go and
\emph{texdir} is where you want the \TeX\ style files to go.

\subsubsection{Installing from SVN}\label{install.svn}
The most current code can be obtained from the svn repository.
It is browsable at \url{https://svn.mathweb.org/repos/LaTeXML}.
Anonymous checkout is available using the command:
\begin{lstlisting}[style=shell]
  svn co https://svn.mathweb.org/repos/LaTeXML
\end{lstlisting}
After checkout, the installation is similar to \ref{install.source}:
\begin{lstlisting}[style=shell]
   cd LaTeXML
   perl Makefile.PL
   make
   make test
   make install
\end{lstlisting}
where the last command would typically be execuded as root.

\emph{Note} that if you have already checked out a copy of \LaTeXML, you can update
and rebuild your copy by using the following
\begin{lstlisting}[style=shell]
   cd LaTeXML
   svn update
   make
   make test
   make install
\end{lstlisting}

\subsection[OS-Specific Notes]{Operating System Specific Notes}\label{install.osnotes}
With \emph{no} implied endorsement of any of these systems.

\subsubsection[RPM-based systems]{RPM-based systems}\label{install.fedora}
For Fedora and recent RedHat-based Enterprise distributions
(Redhat 6, Centos 6 and similar), use one of the following procedures.

\paragraph{Installing prebuilt}\\
\begin{itemize}
\item Download \CurrentFedora
\item Install \LaTeXML, and its prerequisites, using the command:
\begin{lstlisting}[style=shell]
   yum --nogpgcheck localinstall LaTeXML-@\CurrentVersion@-*.ltxml.noarch.rpm
\end{lstlisting}
\end{itemize}
% If you had previously installed \LaTeXML\ rpm, you may want to explicitly
% uninstall it \emph{first}, since the name has changed:
% \begin{lstlisting}[style=shell]
%    yum remove perl-LaTeXML
% \end{lstlisting}

\paragraph{Installing from Source or SVN}
The prerequisites can be installed (or updated) by running this command as root: 
\begin{lstlisting}[style=shell]
yum install perl-Parse-RecDescent \
    perl-XML-LibXML perl-XML-LibXSLT \
    ImageMagick ImageMagick-perl
\end{lstlisting}
Afterwards, you may use the general procedure for
\ref{install.source} or \ref{install.svn}.

\subsubsection[Older Enterprise systems]{Older Enterprise-style RPM-based systems (RedHat, Centos)}\label{install.enterprise}
For older Red Hat Enterprise Linux 5 and derivatives (Centos, Scientific Linux),
we provide two additional packages which are needed.

\paragraph{Installing prebuilt}\\
\begin{itemize}
\item Choose and download the following according to your architecture:
\begin{description}
\item[32bit]
   \href{releases/perl-XML-LibXML-XPathContext-0.07-1.c5.ltxml.i386.rpm}{perl-XML-LibXML-XPathContext},
   \href{releases/perl-XML-LibXSLT-1.58-1.c5.ltxml.i386.rpm}{perl-XML-LibXSLT}
\item[64bit]
   \href{releases/perl-XML-LibXML-XPathContext-0.07-1.c5.ltxml.x86_64.rpm}{perl-XML-LibXML-XPathContext},
   \href{releases/perl-XML-LibXSLT-1.58-1.c5.ltxml.x86_64.rpm}{perl-XML-LibXSLT}
\item[Source RPM]
    \href{releases/perl-XML-LibXML-XPathContext-0.07-1.c5.ltxml.src.rpm}{perl-XML-LibXML-XPathContext},
    \href{releases/perl-XML-LibXSLT-1.58-1.c5.ltxml.src.rpm}{perl-XML-LibXSLT}
\end{description}
\item Download \CurrentCentos
\item Install using the command:
\begin{lstlisting}[style=shell]
   yum --nogpgcheck localinstall LaTeXML-@\CurrentVersion@-*.ltxml.noarch.rpm \
       perl-XML-LibXML-XPathContext-0.07-1.*   \
       perl-XML-LibXSLT-1.58-1.*
\end{lstlisting}
\end{itemize}

\subsubsection{Debian-based systems}\label{install.debian}
\paragraph{Installing prebuilt}
LaTeXML has been included in the Debian repositories (thanks Atsuhito Kohda);
it should be installable using
\begin{lstlisting}[style=shell]
  sudo apt-get install latexml
\end{lstlisting}
This will automatically include any needed dependencies.

\paragraph{Installing from Source or SVN}
The prerequisites can be installed (or updated) by running this command as root: 
\begin{lstlisting}[style=shell]
   sudo apt-get install   \
      libparse-recdescent-perl \
      libxml2 libxml-libxml-perl \
      libxslt1.1 libxml-libxslt-perl  \
      imagemagick perlmagick
\end{lstlisting}
Afterwards, you may use the general procedure for
\ref{install.source} or \ref{install.svn}.

%Some \href{http://rhaptos.org/devblog/reedstrm/latexml}{notes} on installation on Debian
%based systems are also available.

\subsubsection{MacOS}\label{install.macos}
\paragraph{Installing prebuilt}
LaTeXML has been included in the \href{http://www.macports.org}{MacPorts}
repository (thanks Andrew Fernandes);
it should be installable, along with its prerequisites, using the command
\begin{lstlisting}[style=shell]
  sudo port install LaTeXML
\end{lstlisting}

% \paragraph{For the Adventurous}  As an easy alternative --- if it works ---
% download \CurrentMacOS, save in it's own directory as \texttt{Portfile}
% (without the version number) and, within that directory, run
% \begin{lstlisting}[style=shell]
%   sudo port install
% \end{lstlisting}
% This should install \LaTeXML\ and it's all dependencies;
% Otherwise, continue as below.

\paragraph{Installing from Source or SVN}
The prerequisites can be installed by running this command as root: 
\begin{lstlisting}[style=shell]
  sudo port install    \
      p5-xml-libxml p5-xml-libxslt  \
      p5-parse-recdescent p5-perlmagick
\end{lstlisting}
Afterwards, you may use the general procedure for
\ref{install.source} or \ref{install.svn}.

\emph{Note} there have been issues reported regarding \verb|DB_File|
not being installed;  Apparently you must install the 
the db `variant' of perl, rather than the gdbm variant;
that is, you must run \verb|sudo port install perl +db|
(possibly after uninstalling perl first?).

\subsubsection{Windows}\label{install.windows}
There is currently no prebuilt \LaTeXML\ for Windows,
but it does run under \href{http://strawberryperl.com}{Strawberry Perl}
which comes with some of our prerequisites pre-installed,
and provides other needed commands (\texttt{perl}, \texttt{cpan}, \texttt{dmake}).
You may install \LaTeXML\ either by \ref{install.windows.source}
or \ref{install.windows.svn} after \ref{install.windows.prerequisites}.

\paragraph{Installing prerequisites}\label{install.windows.prerequisites}\\
\begin{itemize}
\item If you don't have Strawberry Perl already installed:\\
   Follow the instructions at \href{http://strawberryperl.com}{Strawberry Perl}
   to download and install the latest version of Strawberry Perl\\
\item If you don't have Ghostscript already installed:\\
   Follow the instructions at \href{http://sourceforge.net/projects/ghostscript/}{Ghostscript}
   to download and install the latest version of Ghostscript program
   (needed by Image::Magick)
\item If you don't have Imagemagick already installed:\\
   Follow the instructions at
   \href{http://www.imagemagick.org/script/binary-releases.php#windows}{Image Magick}
   to download and install the latest ImageMagick binary.
\item Install the \texttt{Parse::RecDescent} Perl module by typing the following in the command prompt
\begin{lstlisting}[style=shell]
  cpan -i Parse::RecDescent
\end{lstlisting}
\item Install the \texttt{Image::Magick} Perl module using the command
\begin{lstlisting}[style=shell]
  cpan -i Image::Magick
\end{lstlisting}
 If the install fails with a single error on the write.t test, don't worry,
it is a well known error on Win32 systems that is irrelevant.
In that case, run cpan and then in its interactive shell,
proceed to force the install
\begin{lstlisting}[style=shell]
  cpan
  cpan> force install Image::Magick
\end{lstlisting}
\end{itemize}

\paragraph{Installing from Source}\label{install.windows.source}\\
After \ref{install.windows.prerequisites},
you my install the released source of \LaTeXML,
by following the general procedure but using Strawberry Perl's \texttt{dmake} command:
\begin{itemize}
\item Download \CurrentTarball\\
\item Unpack and build \LaTeXML:
\begin{lstlisting}[style=shell]
 tar zxvf LaTeXML-@\CurrentVersion@.tar.gz\
 cd LaTeXML-@\CurrentVersion@
 perl Makefile.PL
 dmake
 dmake test
 dmake install
\end{lstlisting}
\end{itemize}

\paragraph{Installing from SVN}\label{install.windows.svn}\\
Alternatively, after \ref{install.windows.prerequisites},
you may install a very current version of \LaTeXML\ from SVN:
\begin{itemize}
\item If you don't already have an SVN client,
   follow the instructions at \href{http://tortoisesvn.net/}{TortoiseSVN}
   to download and install the most recent version.
\item Fetch \LaTeXML\ from the repository and build:
\begin{lstlisting}[style=shell]
 svn co https://svn.mathweb.org/repos/LaTeXML
 cd LaTeXML
 perl Makefile.PL
 dmake
 dmake test
 dmake install
\end{lstlisting}
\end{itemize}

\emph{Note} that if you have already checked out a copy of \LaTeXML, you can update
and rebuild your copy by using the following
\begin{lstlisting}[style=shell]
   cd LaTeXML
   svn update
   dmake
   dmake test
   dmake install
\end{lstlisting}

% Many of the prerequisite packages are available in
% ppm form (for \href{http://activestate.com/}{ActivePerl} (I think).
% I will attempt to generate a ppm for LaTeXML in the near
% future, depending on user interest --- and assistance!

\subsection{Prerequisites}\label{prerequisites}
The following list gives an overview of the Perl modules required by LaTeXML;
As a general rule, \LaTeXML\ will run with any version released within
the last several years.
% %3A%3A for :: ???
\begin{description}
\item[\href{http://search.cpan.org/search?query=Parse::RecDescent&mode=module}{Parse::RecDescent}]
    a useful grammar based parser module.
\item[\href{http://search.cpan.org/search?query=Image::Magick&mode=module}{Image::Magick}]
    provides bindings to the \href{http://www.imagemagick.org/}{ImageMagick} library.
\item[\href{http://search.cpan.org/search?query=XML::LibXML&mode=module}{XML::LibXML}]
    provides bindings to the C library libxml2
    (available from \href{http://www.xmlsoft.org}{xmlsoft}).
    Versions before 1.61 will require the additional \texttt{XML::LibXML::XPathContext} module.
\item[\href{http://search.cpan.org/search?query=XML::LibXSLT&mode=module}{XML::LibXSLT}]
    provides bindings to the C library libxslt
    (from \href{http://www.xmlsoft.org}{xmlsoft},
%  <li><tt><a href="http://search.cpan.org/search?query=XML::LibXML::XPathContext&mode=module">XML::LibXML::XPathContext</a></tt>
%      Depends on the XML::LibXML module.
\item[\href{http://search.cpan.org/search?query=DB_File&mode=module}{DB\_File}]
    usually part of a standard Perl installation, provided
    \href{http://www.sleepycat.com}{BerkeleyDB} is installed.
\item[\href{http://search.cpan.org/search?query=Test::Simple&mode=module}{Test::Simple}]
    usually part of a standard Perl installation.
\end{description}

\subsection{License}\label{license}
As this software was developed as part of work done by the
United States Government, it is not subject to copyright,
and is in the public domain.
Note that according to
\href{http://www.gnu.org/licences/license-list.html#PublicDomain}{Gnu.org}
public domain is compatible with GPL.

\subsection{Archived Releases:}\label{archive}
\AllReleases.

%============================================================
\section{Documentation}\label{docs}
If you're lucky, all that should be needed to convert
a \TeX\ file, \textit{mydoc}\texttt{.tex} to XML, and
then to XHTML+MathML would be:
\begin{lstlisting}[style=shell]
   latexml --dest=\textit{mydoc}.xml \textit{mydoc}
   latexmlpost -dest=\textit{somewhere/mydoc}.xhtml \textit{mydoc}.xml
\end{lstlisting}
This will carry out a default transformation into XHTML+MathML.  If you
give the destination extension with html, it will generate HTML+images.

If you're not so lucky, or want to get fancy, well \ldots dig deeper:
\begin{description}
\item[\href{manual/index.xhtml}{LaTeXML Manual}]
    Overview of LaTeXML (\href{manual.pdf}{\PDFIcon}).
\item[\href{manual/commands/latexml.xhtml}{\texttt{latexml}}]
    describes the \texttt{latexml} command.
\item[\href{manual/commands/latexmlpost.xhtml}{\texttt{latexmlpost} command}]
   describes the \texttt{latexmlpost} command for postprocessing.
\end{description}

% Possibly, eventually, want to expose:
%   http://www.mathweb.org/wiki/????
% But, it doesn't have anything in it yet.

%============================================================
\section{Contacts \& Support}\label{contact}

\paragraph{Mailing List}
There is a low-volume mailing list for questions, support and comments.
See \href{http://lists.jacobs-university.de/mailman/listinfo/project-latexml}{\texttt{latexml-project}} for subscription information.


\paragraph{Bug-Tracker}
  There is a Trac bug-tracking system for reporting bugs, or checking the
  status of previously reported bugs at
 \href{https://trac.mathweb.org/LaTeXML/}{Bug-Tracker}.


\paragraph{Thanks} to our friends at
the \href{http://kwarc.info}{KWARC Research Group}
for hosting the mailing list, Trac system and svn repository,
as well as general moral support.
Thanks also to \href{http://nist.gov/sima}{Systems Integration for Manufacturing Applications}
for funding portions of the research and development.

\paragraph{Author} \href{mailto:bruce.miller@nist.gov}{Bruce Miller}.
%============================================================
\section{Privacy Notice}\label{privacy}
We adhere to \href{http://www.nist.gov/public_affairs/privacy.htm}{NIST's Privacy, Security and Accessibility Policy}.
%============================================================

\end{document}
