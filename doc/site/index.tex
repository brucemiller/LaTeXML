\documentclass{article}
\usepackage{latexml}
\usepackage{hyperref}
\usepackage{../sty/latexmldoc}
\usepackage{listings}
% Should the additional keywords be indexed?
\lstdefinestyle{shell}{language=bash,escapechar=@,basicstyle=\ttfamily\small,%
   morekeywords={latexml,latexmlpost,latexmlmath},
   moredelim=[is][\itshape]{\%}{\%}}

\input{releases.tex}
\newcommand{\PDFIcon}{\includegraphics{pdf}}

\title{\LaTeXML\ \emph{A \LaTeX\ to XML Converter}}
\lxKeywords{LaTeXML, LaTeX to XML, LaTeX to HTML, LaTeX to MathML, LaTeX to ePub, converter}
%============================================================
\begin{lxNavbar}
\lxRef{top}{\includegraphics{../graphics/latexml}}\\
\includegraphics{../graphics/mascot}\\
\lxContextTOC\\
% Direct link to manual from navbar
\vspace{1cm}
\URL[\hspace{4em}\LARGE The\ Manual]{./manual/}
\end{lxNavbar}
%============================================================

\begin{document}
\label{top}
\maketitle

%============================================================
The \emph{Current release} is \htmlref{\CurrentVersion}{get},
but it is \emph{terribly} out of date!
Please consider alpha versions 0.7.9 (see \ref{source.git}).
Release 0.8.0 is on it's way!

In the process of developing the
\href{http://dlmf.nist.gov/}{Digital Library of Mathematical Functions},
we needed a means of transforming
the \LaTeX\ sources of our material into XML which would be used
for further manipulations, rearrangements and construction of the web site.
In particular, a true `Digital Library' should focus on the \emph{semantics}
of the material, and so we should convert the mathematical material into both
content and presentation MathML.
At the time, we found no software suitable to our needs, so we began
development of \LaTeXML\ in-house.  

In brief, \texttt{latexml} is a program, written in Perl, that attempts to
faithfully mimic \TeX's behaviour, but produces XML instead of dvi.
The document model of the target XML makes explicit the model implied
by \LaTeX.
The processing and model are both extensible; you can define
the mapping between \TeX\ constructs and the XML fragments to be created.
A postprocessor, \texttt{latexmlpost} converts this
XML into other formats such as HTML or XHTML, with options
to convert the math into MathML (currently only presentation) or images.

\emph{Caveats}: It isn't finished, there are gaps in the coverage,
particularly in missing implementations of the many useful \LaTeX\ packages.
But is beginning to stabilize and interested parties
are invited to try it out, give feedback and even to help out.


%============================================================
\section{Examples}\label{examples}\index{examples}
At the moment, the best example of \LaTeXML's output is 
the \href{http://dlmf.nist.gov/}{DLMF} itself.
There is, of course, a fair amount of insider, special case,
code, but it shows what can be done.

Some highlights:
\begin{description}
\item[\href{examples/tabular/tabular.html}{LaTeX tabular}]
    from the \LaTeX\ manual, p.205.
    (\href{examples/tabular/tabular.tex}{\TeX},
     \href{examples/tabular/tabular.pdf}{\PDFIcon})
\item[\url{http://latexml.mathweb.org/editor}] an online editor/showcase
  of things that \LaTeXML\ can do.
\item[\url{http://arxmliv.kwarc.info}] An experiment processing
  the entire \url{http://arXiv.org}.
\end{description}
And, of course
\begin{description}
\item[\href{http://dlmf.nist.gov/}{DLMF}]
   The Digital Library of Mathematical Functions was the
   primary instigator for this project.
\item[\href{manual/}{\LaTeXML\ Manual}]
   The \LaTeXML\ User's manual (\href{manual.pdf}{\PDFIcon}).
\item[And these pages] were produced using \LaTeXML, as well.
\end{description}

%============================================================
\section{Get \LaTeXML}\label{get}\index{get}
\paragraph{Current Release:}\label{download.current}
The current release is \htmlref{\CurrentVersion}{get},
but it is \emph{terribly} out of date!
Release 0.8.0 is on it's way!

\emph{Normally}, it is preferable to install a platform-specific, prebuilt release.
By using the appropriate, system-specific, \textit{magiccommand}
both \LaTeXML\ and any prerequisites
(additional perl modules that it needs, some of which themselves depend on C libraries)
will be installed automatically.
\begin{lstlisting}[style=shell]
%magiccommand% LaTeXML
\end{lstlisting}
See instructions for specific systems, below.

But as the currrent release is so out-of-date,
please consider alpha versions 0.7.9 (see \ref{source.git}),
in which case you'll need to pre-install the prerequisites
by following the instructions at \ref{install.general}.

Also see the \href{Changes}{Change Log}.

\subsection{General Installation Procedures}\label{install.general}\index{install}
\subsubsection{Installing Prerequisites}\label{install.prerequisites}
If you cannot install a prebuilt package, or want to use the current
pre-release version from the repository, you will need to pre-install some or
all of the prerequisites.

The easiest route is using the appropriate, system-specific,
\textit{magiccommand} to install the prerequisites
\begin{lstlisting}[style=shell]
%magiccommand% %prerequisite% ...
\end{lstlisting}
See instructions for specific systems, below.

To install the prerequisites by hand, the following instructions may help:
\begin{itemize}
\item If \texttt{libxml2} and \texttt{libxslt} are are not already installed:\\
  Follow the instructions at \href{http://www.xmlsoft.org}{XMLSoft} to
  download and install the most recent versions of \texttt{libxml2} and \texttt{libxslt}.
\item Use CPAN to install the required perl modules (typically as root), something like this
\end{itemize}
\begin{lstlisting}[style=shell]
perl -MCPAN -e shell
  cpan> install Archive::Zip, File::Which, Image::Size, IO::String
  cpan> install libwww, JSON::XS, LWP, Parse::RecDescent, URI
  cpan> install XML::LibXML, XML::LibXSLT
  cpan> quit
\end{lstlisting}

\subsubsection{Installing Optional Prerequisites}\label{install.optional}
The following packages are usually desirable, but aren't required to run \LaTeXML.
\begin{description}
\item[\TeX] Most users of \LaTeXML\ will want to install \TeX, as well.  \LaTeXML\ will
use \TeX's style files directly in some cases, providing broader coverage.

Moreover, if \TeX\ is present when \LaTeXML\ is being installed, it will install a couple of
its own for use with regular \TeX, or \LaTeX\ runs;
So if you are going to install \TeX, install it first!

\item[Image::Magick] provides a handy library of image manipulation routines.
When they are present \LaTeXML\ is able to carry out more image processessing,
such as transformations by the \texttt{graphicx} package, and conversion of math to images.
When the package is not present, some such operations will not be supported.

Unfortunately, it is sometimes a difficult package to install.
Follow the instructions at \href{http://www.imagemagick.org/}{ImageMagick}
to download and install the latest version of ImageMagick being sure to enable
and build the Perl binding along with it.

\item[Graphics::Magick] is an \emph{alternative} to \texttt{Image::Magick} that \LaTeXML\ will
use if is found on the system; it may (or may not ) be easier to install, although it
is less widespread.
\end{description}
\emph{Note to packagers:} If you are preparing a compiled installation package (such as rpm or deb) for
\LaTeXML, and the above packages are easily installable in your distribution,
you probably should include them as dependencies of \LaTeXML.

\subsubsection{Getting the source}\label{getting.source}
\paragraph{Source tarball}\label{source.tarball}
You can either download a tar file containing the source:
\begin{itemize}
\item Download \CurrentTarball
\item Unpack and Build \LaTeXML, following the usual Perl module incantations:
\begin{lstlisting}[style=shell]
tar zxvf LaTeXML-@\CurrentVersion@.tar.gz
cd LaTeXML-@\CurrentVersion@
\end{lstlisting}
\end{itemize}

\paragraph{git checkout}\label{source.git}
Or you can fetch the most current code from github by using the \texttt{git} command:
\begin{lstlisting}[style=shell]
git clone https://github.com/brucemiller/LaTeXML.git
\end{lstlisting}
You can also browser the current code at \url{https://github.com/brucemiller/LaTeXML/}.

[Once you've fetched the code from you github, you can easily keep up-to-date
by ocassionally running the command:
\begin{lstlisting}[style=shell]
git pull
\end{lstlisting}
in the source directory, and then repeating the following building and install commands.]

\subsubsection{Building}\label{build.source}
Whether you've downloaded using a tar file, or from git,
you build the system using the standard Perl procedure:
\begin{lstlisting}[style=shell]
cd LaTeXML  [@or@  LaTeXML-@\CurrentVersion@]
perl Makefile.PL
make
make test
\end{lstlisting}
The last step runs the system through some basic tests,
which should complete without error (although some tests may be `skipped'
under certain circumstances).
[On windows systems, use \texttt{dmake} instead of \texttt{make}.]

\emph{Note:} You can specify nonstandard place to install files
--- and possibly avoid the need to install as root! ---
by modifying the Makefile creating command above to
\begin{lstlisting}[style=shell]
perl Makefile.PL PREFIX=%perldir% TEXMF=%texdir%
\end{lstlisting}
where \emph{perldir} is where you want the perl related files to go and
\emph{texdir} is where you want the \TeX\ style files to go.
(See \texttt{perl perlmodinstall} for more details and options.)

\subsubsection{Installing}\label{install.source}
After building, you will finally want to install \LaTeXML\
where the OS can find the files.  You'll typically need to be root:
\begin{lstlisting}[style=shell]
su
make install
\end{lstlisting}
or perhaps
\begin{lstlisting}[style=shell]
sudo make install
\end{lstlisting}

\subsection[OS-Specific Notes]{Operating System Specific Notes}\label{install.osnotes}
With \emph{no} implied endorsement of any of these systems.

\subsubsection[RPM-based systems]{RPM-based systems}\label{install.fedora}\index{fedora}
For Fedora and recent RedHat-based Enterprise distributions
(Redhat 6, Centos 6, Scientific Linux 6 and similar),
most software is obtained and installed via the yum repository.

\paragraph{Installing prebuilt}\\
\begin{itemize}
\item Download \CurrentFedora
\item Install \LaTeXML, and its prerequisites, using the command:
\begin{lstlisting}[style=shell]
yum --nogpgcheck localinstall LaTeXML-@\CurrentVersion@-*.ltxml.noarch.rpm
\end{lstlisting}
\end{itemize}
Adding \texttt{ImageMagick ImageMagick-perl} will include the optional ImageMagick.

\paragraph{Installing only prerequisites}
The prerequisites can be installed (or updated) by running this command as root: 
\begin{lstlisting}[style=shell]
yum install \
  perl-Archive-Zip perl-File-Which perl-Image-Size \
  perl-IO-string perl-JSON-XS perl-libwww-perl perl-Parse-RecDescent \
  perl-URI perl-XML-LibXML perl-XML-LibXSLT
\end{lstlisting}
Install the optional packages with:
\begin{lstlisting}[style=shell]
yum install texlive ImageMagick ImageMagick-perl
\end{lstlisting}
Then continue by using the general procedure from \ref{build.source}.

% Note that if the command to make the Makefile fails with
% ``Can't locate ExtUtils/MakeMaker.pm\ldots'' or warns that Test::Simple is missing,
% you will need to install them as follows:
% \begin{lstlisting}[style=shell]
% yum install perl-ExtUtils-MakeMaker perl-Test-Simple
% \end{lstlisting}


% \subsubsection[Older Enterprise systems]{Older Enterprise-style RPM-based systems (RedHat, Centos)}\label{install.enterprise}\index{redhat}\index{centos}
% For older Red Hat Enterprise Linux 5 and derivatives (Centos, Scientific Linux),
% we provide two additional packages which are needed.

% \paragraph{Installing prebuilt}\\
% \begin{itemize}
% \item Choose and download the following according to your architecture:
% \begin{description}
% \item[32bit]
%    \href{releases/perl-XML-LibXML-XPathContext-0.07-1.c5.ltxml.i386.rpm}{perl-XML-LibXML-XPathContext},
%    \href{releases/perl-XML-LibXSLT-1.58-1.c5.ltxml.i386.rpm}{perl-XML-LibXSLT}
% \item[64bit]
%    \href{releases/perl-XML-LibXML-XPathContext-0.07-1.c5.ltxml.x86_64.rpm}{perl-XML-LibXML-XPathContext},
%    \href{releases/perl-XML-LibXSLT-1.58-1.c5.ltxml.x86_64.rpm}{perl-XML-LibXSLT}
% \item[Source RPM]
%     \href{releases/perl-XML-LibXML-XPathContext-0.07-1.c5.ltxml.src.rpm}{perl-XML-LibXML-XPathContext},
%     \href{releases/perl-XML-LibXSLT-1.58-1.c5.ltxml.src.rpm}{perl-XML-LibXSLT}
% \end{description}
% \item Download \CurrentCentos
% \item Install using the command:
% \begin{lstlisting}[style=shell]
%    yum --nogpgcheck localinstall LaTeXML-@\CurrentVersion@-*.ltxml.noarch.rpm \
%        perl-XML-LibXML-XPathContext-0.07-1.*   \
%        perl-XML-LibXSLT-1.58-1.*
% \end{lstlisting}
% \end{itemize}

\subsubsection{Debian-based systems}\label{install.debian}\index{debian}
For Debian-based systems (including Ubuntu), the deb repositories
are generally used for software installation.

\paragraph{Installing prebuilt}
\LaTeXML\ has been included in the Debian repositories (thanks Atsuhito Kohda);
it, along with dependencies, should be installable using
\begin{lstlisting}[style=shell]
sudo apt-get install latexml
\end{lstlisting}
Adding \texttt{imagemagick perlmagick} will include the optional ImageMagick.

\paragraph{Installing only prerequisites}
The prerequisites can be installed (or updated) by running this command as root: 
\begin{lstlisting}[style=shell]
sudo apt-get install   \
  libarchive-zip-perl libfile-which-perl libimage-size-perl \
  libio-string-perl libjson-xs-perl libwww-perl libparse-recdescent-perl \
  liburi-perl libxml2 libxml-libxml-perl libxslt1.1 libxml-libxslt-perl
\end{lstlisting}
Install the optional packages with:
\begin{lstlisting}[style=shell]
sudo apt-get install texlive imagemagick perlmagick 
\end{lstlisting}
Then continue by using the general procedure from \ref{build.source}.

%Some \href{http://rhaptos.org/devblog/reedstrm/latexml}{notes} on installation on Debian
%based systems are also available.

\subsubsection{MacOS}\label{install.macos}\index{macintosh}
The \href{http://www.macports.org}{MacPorts} tool provides a handy
repository for software installation on Mac computers;
download and install macports from that site.
Then, use one of the following procedures depending on which version
of \LaTeXML\ you wish to install:

\paragraph{Installing prebuilt}
\LaTeXML\ has been included in the \href{http://www.macports.org}{MacPorts}
repository (thanks Andrew Fernandes);
it should be installable, along with its prerequisites, using the command
\begin{lstlisting}[style=shell]
sudo port install LaTeXML
\end{lstlisting}
Adding \texttt{p5-perlmagick} will include the optional ImageMagick.

% \paragraph{For the Adventurous}  As an easy alternative --- if it works ---
% download \CurrentMacOS, save in it's own directory as \texttt{Portfile}
% (without the version number) and, within that directory, run
% \begin{lstlisting}[style=shell]
%   sudo port install
% \end{lstlisting}
% This should install \LaTeXML\ and it's all dependencies;
% Otherwise, continue as below.

\paragraph{Installing only prerequisites}
The prerequisites can be installed by running this command as root: 
\begin{lstlisting}[style=shell]
sudo port install \
  p5-archive-zip p5-file-which p5-image-size p5-io-string \
  p5-json-xs p5-libwww-perl p5-parse-recdescent p5-uri \
  p5-xml-libxml p5-xml-libxslt
\end{lstlisting}
Install the optional packages with:
\begin{lstlisting}[style=shell]
sudo port install p5-perlmagick texlive
\end{lstlisting}
although you may prefer other \TeX\ systems on the Mac.
Then continue by using the general procedure from \ref{build.source}.

% \emph{Note:} There have been issues reported regarding \verb|DB_File|
% not being installed;  Apparently you must install the 
% the db `variant' of perl, rather than the gdbm variant;
% that is, you must run \verb|sudo port install perl +db|
% (possibly after uninstalling perl first?).

\emph{Note:} There have been issues reported with recent
installations of Perl with MacPorts:  it will install \LaTeXML's
executables in a directory specific to the current version of Perl
(eg. \texttt{/opt/local/libexec/perl5.12/sitebin},
instead of \texttt{/opt/local/bin} which would be in your \texttt{PATH} environment variable).
Apparently this is a feature, not a bug; it only happens when installing from source or git;
not when installing the MacPorts port.  There are three workarounds, each with disadvantages:
\begin{itemize}
\item Watch for where the scripts get installed and that directory to your \texttt{PATH}
  environment variable;
\item Set up symbolic links from a directory in your path, such as \texttt{/opt/local/bin},
  to the actual installed locations;
\item Use the makefile options to choose an installation directory (see \ref{install.options}):
\begin{lstlisting}[style=shell]
perl Makefile.PL INSTALLSITEBIN=/opt/local/bin INSTALLSITESCRIPT=/opt/local/bin
\end{lstlisting}
\end{itemize}

\subsubsection{Windows}\label{install.windows}\index{windows}
There is currently no prebuilt \LaTeXML\ for Windows,
but it does run under \href{http://strawberryperl.com}{Strawberry Perl},
which comes with \emph{many} of our prerequisites pre-installed,
and provides other needed commands (\texttt{perl}, \texttt{cpan}, \texttt{dmake}).

\paragraph{Installing prerequisites}\label{install.windows.prerequisites}\\
\begin{itemize}
\item If you want to install a \TeX\ system, do so first.
\item If you don't have Strawberry Perl already installed:\\
   Follow the instructions at \href{http://strawberryperl.com}{Strawberry Perl}
   to download and install the latest version of Strawberry Perl\\
% \item If you don't have Ghostscript already installed:\\
%    Follow the instructions at \href{http://sourceforge.net/projects/ghostscript/}{Ghostscript}
%    to download and install the latest version of Ghostscript program
%    (needed by Image::Magick)
% \item If you don't have Imagemagick already installed:\\
%    Follow the instructions at
%    \href{http://www.imagemagick.org/script/binary-releases.php#windows}{Image Magick}
%    to download and install the latest ImageMagick binary.
\item Install other prerequisite modules by typing the following in the command prompt
\begin{lstlisting}[style=shell]
cpan -i Image::Size
cpan -i Parse::RecDescent
\end{lstlisting}
% \item Install the \texttt{Image::Magick} Perl module using the command
% \begin{lstlisting}[style=shell]
% cpan -i Image::Magick
% \end{lstlisting}
%  If the install fails with a single error on the write.t test, don't worry,
% it is a well known error on Win32 systems that is irrelevant.
% In that case, run cpan and then in its interactive shell,
% proceed to force the install
% \begin{lstlisting}[style=shell]
% cpan
% cpan> force install Image::Magick
% \end{lstlisting}
\end{itemize}
Installing the optional package \texttt{Image::Magick} on Windows seems to be problematic;
you're on your own, there!
Or you may  have better luck with \texttt{Graphics::Magick}.

Then continue by using the general procedure from \ref{build.source},
but note that the \texttt{dmake} command should be used in place of \texttt{make}.

\subsection{Archived Releases:}\label{archive}
\AllReleases.

%============================================================
\section{Documentation}\label{docs}
If you're lucky, all that should be needed to convert
a \TeX\ file, \textit{mydoc}\texttt{.tex} to XML, and
then to XHTML+MathML would be:
\begin{lstlisting}[style=shell]
   latexml --dest=%mydoc%.xml %mydoc%
   latexmlpost -dest=%somewhere/mydoc%.xhtml %mydoc%.xml
\end{lstlisting}
This will carry out a default transformation into XHTML+MathML.  If you
give the destination extension with html, it will generate HTML+images.

If you're not so lucky, or want to get fancy, well \ldots dig deeper:
\begin{description}
\item[\href{manual/}{LaTeXML Manual}]
    Overview of LaTeXML (\href{manual.pdf}{\PDFIcon}).
\item[\href{manual/commands/latexml.html}{\texttt{latexml}}]
    describes the \texttt{latexml} command.
\item[\href{manual/commands/latexmlpost.html}{\texttt{latexmlpost} command}]
   describes the \texttt{latexmlpost} command for postprocessing.
\end{description}

% Possibly, eventually, want to expose:
%   http://www.mathweb.org/wiki/????
% But, it doesn't have anything in it yet.

%============================================================
\section{Contacts \& Support}\label{contact}

\paragraph{Mailing List}
There is a low-volume mailing list for questions, support and comments.
See \href{http://lists.jacobs-university.de/mailman/listinfo/project-latexml}{\texttt{latexml-project}} for subscription information.


\paragraph{Bug-Tracker}
We are using the git repository hosted at
\href{https://github.com/brucemiller/LaTeXML/}{https://github.com/brucemiller/LaTeXML/}.
You can browse the code, the latest changes, and check-out the current code from
there (see \ref{get}).  There is also an Issues database where you can
file bug reports or feature requests.

%  There is a Trac bug-tracking system for reporting bugs, or checking the
%  status of previously reported bugs at
%  \href{https://trac.mathweb.org/LaTeXML/}{Bug-Tracker}.

% To report bugs, please:
% \begin{itemize}
% \item \href{http://trac.mathweb.org/register/register}{Register} a Trac account
%   (preferably give an email so that you'll get notifications about activity regarding the bug).
% \item \href{http://trac.mathweb.org/LaTeXML/newticket}{Create a ticket}
% \end{itemize} 

\paragraph{Thanks} to our friends at
the \href{http://kwarc.info}{KWARC Research Group}
for hosting the mailing list, the original Trac system and svn repository,
as well as general moral support.

%%Thanks also to \href{http://www.nist.gov/el/msid/sima/}{Systems Integration for Manufacturing Applications}
%%for funding portions of the research and development.

\paragraph{Author} \href{mailto:bruce.miller@nist.gov}{Bruce Miller}.
%============================================================
\section{Licence \& Notices}\label{notices}

\paragraph{Licence}
The research software provided on this web site (``software'') is
provided by NIST as a public service. You may use, copy and distribute
copies of the software in any medium, provided that you keep intact
this entire notice. You may improve, modify and create derivative
works of the software or any portion of the software, and you may copy
and distribute such modifications or works. Modified works should
carry a notice stating that you changed the software and should note
the date and nature of any such change. Please explicitly acknowledge
the National Institute of Standards and Technology as the source of
the software.

The software was developed by NIST employees. NIST employee
contributions are not subject to copyright protection within the
United States.

The software is thus released into the Public Domain.

Note that according to
\href{http://www.gnu.org/licences/license-list.html#PublicDomain}{Gnu.org}
public domain is compatible with GPL.

\paragraph{Disclaimer}
The software is expressly provided ``AS IS.'' NIST MAKES NO WARRANTY OF
ANY KIND, EXPRESS, IMPLIED, IN FACT OR ARISING BY OPERATION OF LAW,
INCLUDING, WITHOUT LIMITATION, THE IMPLIED WARRANTY OF
MERCHANTABILITY, FITNESS FOR A PARTICULAR PURPOSE, NON-INFRINGEMENT
AND DATA ACCURACY. NIST NEITHER REPRESENTS NOR WARRANTS THAT THE
OPERATION OF THE SOFTWARE WILL BE UNINTERRUPTED OR ERROR-FREE, OR THAT
ANY DEFECTS WILL BE CORRECTED. NIST DOES NOT WARRANT OR MAKE ANY
REPRESENTATIONS REGARDING THE USE OF THE SOFTWARE OR THE RESULTS
THEREOF, INCLUDING BUT NOT LIMITED TO THE CORRECTNESS, ACCURACY,
RELIABILITY, OR USEFULNESS OF THE SOFTWARE.

You are solely responsible for determining the appropriateness of
using and distributing the software and you assume all risks
associated with its use, including but not limited to the risks and
costs of program errors, compliance with applicable laws, damage to or
loss of data, programs or equipment, and the unavailability or
interruption of operation. This software is not intended to be used in
any situation where a failure could cause risk of injury or damage to
property.

\paragraph{Privacy Notice}
We adhere to \href{http://www.nist.gov/public_affairs/privacy.cfm}{NIST's Privacy, Security and Accessibility Policy}.
%============================================================

\end{document}
